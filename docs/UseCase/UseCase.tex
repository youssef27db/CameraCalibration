\documentclass[12pt]{scrartcl}
\usepackage[utf8]{inputenc}
\usepackage[ngerman]{babel}
\usepackage[T1]{fontenc}
\usepackage{lmodern}
\usepackage{amsmath, amssymb}
\usepackage{svg}
\usepackage{graphicx}
\usepackage{pdfpages}
\usepackage{minted}
\usepackage[colorlinks=true, linkcolor=black, urlcolor=blue]{hyperref}
\usepackage{geometry}
\geometry{a4paper, margin=1in}

\title{\vspace{5cm}\textbf{Use Case:} data-driven camera calibration for light field camera systems}
\author{\textbf{Author: } Youssef Daoudi El Boukhrissi (4277022)}
\date{\today}

\begin{document}
\maketitle

\newpage
\pagenumbering{arabic}

\section*{Use Case: Cross-Shaped Camera Array 
\newline
[Center + North/East/South/West]}


\subsection*{Phase 1 — Initial Lab Calibration (once)}
\begin{itemize}
  \item \textbf{Option A (target-based, OpenCV):} Checkerboard-based initialization to estimate intrinsic parameters.
  \item \textbf{Option B (target-free, LiFCal):} Sequence-based initialization without a calibration target, estimating    plenoptic intrinsics directly from image observations.
  \item \textbf{Global refinement:} Joint bundle adjustment over the collected frames/image points to minimize reprojection error and obtain a consistent calibration.
  \item \textbf{Baseline:} Store the resulting calibration as the known-good reference for later comparison.
\end{itemize}

\subsection*{Phase 2 — In-Field Self-Check (periodic or event-based)}
\begin{itemize}
  \item \textbf{Triggers:} fixed interval (e.g., hourly), temperature jumps, restart, or detected shock.
  \item \textbf{Short capture window:} record a brief sequence during normal operation (no targets).
  \item \textbf{Option A — Model-based check (baseline comparison):}
    \begin{itemize}
      \item \emph{Windowed reprojection error:} median over the latest frames/micro-images.
      \item \emph{Parameter estimation error:} deviation of key parameters (principal point, effective focal length, low-order distortion) from the baseline/reference calibration.
      \item \emph{Parameter drift:} normalized change of these parameters across consecutive windows.
    \end{itemize}
  \item \textbf{Option B — Feature-based check (persistent tracks):}
    \begin{itemize}
      \item Detect one or a few stable features (e.g., corners/patches) visible across the short window; track with a robust matcher.
      \item \emph{Track-consistency error:} difference between observed feature motion and the motion predicted by the baseline calibration.
    \end{itemize}
  \item \textbf{Decision:} Trigger online recalibration if the Option~A metrics exceed thresholds for $\geq 3$ consecutive windows
        \emph{or} if the Option~B track-consistency error persists above threshold over the same period.
\end{itemize}

\subsection*{Phase 3 — Target-Free Online Recalibration}
\begin{itemize}
  \item \textbf{Trigger:} Start only when the health score has stayed below a threshold across multiple windows. When the health score stays $\leq 70$ across $\geq 3$ consecutive windows (e.g., 20s each).
  \item \textbf{Short data window:} Use a brief, recent multi-camera sequence from normal operation (no targets).
  \item \textbf{What to update:} Only adjust drift-sensitive internal parameters (principal point; small low-order distortion if needed). Keep the rig geometry fixed (no baseline changes).
  \item \textbf{Method (multi-camera BA):} Run bundle adjustment on the latest frames so that corresponding scene points align consistently in all views (i.e., lower reprojection error). No checkerboard required.
  \item \textbf{Commit rule:} Apply the update only if health metrics clearly improve; otherwise defer and retry later.
\end{itemize}

\subsection*{Extras (optional but recommended)}
\begin{itemize}
  \item \textbf{Rollback \& versioning:} keep the last known-good calibration and auto-rollback if a new update worsens the health metrics.
\end{itemize}

\end{document}
