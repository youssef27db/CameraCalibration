\documentclass[12pt]{scrartcl}
\usepackage[utf8]{inputenc}
\usepackage[ngerman]{babel}
\usepackage[T1]{fontenc}
\usepackage{lmodern}
\usepackage{amsmath, amssymb}
\usepackage{svg}
\usepackage{graphicx}
\usepackage{pdfpages}
\usepackage{minted}
\usepackage[colorlinks=true, linkcolor=black, urlcolor=blue]{hyperref}
\usepackage{geometry}
\geometry{a4paper, margin=1in}

\title{\vspace{5cm}\textbf{Requirements:} data-driven camera calibration for light field camera systems}
\author{\textbf{Author: } Youssef Daoudi El Boukhrissi (4277022)}
\date{\today}

\begin{document}
\maketitle

\newpage
\pagenumbering{arabic}

\section*{Requirements for the Cross-Shaped Camera Array Calibration}

Based on the described use case and the methods presented in the reference paper
(LiFCal),
the following requirements define the functional scope and measurable performance goals
for the implementation in this Bachelor’s thesis.

\subsection*{Functional Requirements}
\begin{itemize}
  \item \textbf{FR1 - MLA Initial Calibration:} 
  The system shall perform an initial calibration for all cameras in the cross-shaped array
  using standard OpenCV procedures (checkerboard-based intrinsic and extrinsic estimation).
  \item \textbf{FR2 – Joint Optimization (Bundle Adjustment):} 
  The system shall refine the camera parameters through a multi-camera bundle adjustment (based on the GitHub-Repo (LiFCal))
  minimizing the reprojection error across all overlapping fields of view.
  \item \textbf{FR3 – Periodic Health Monitoring:} 
  The system shall continuously evaluate the calibration quality during operation by computing
  a health score based on reprojection error, parameter estimation error, and parameter drift.
  \item \textbf{FR4 – Automatic Trigger for Recalibration:}
  When the health score remains $\leq70$ over $\geq3$ consecutive time windows,
  the system shall initiate a target-free online recalibration using recent multi-camera data.
  \item \textbf{FR5 – Selective Parameter Update:}
  The recalibration process shall adjust only internal drift-sensitive parameters
  (principal point, possibly low-order distortion) while keeping the rig geometry fixed.
  \item \textbf{FR6 – Rollback Mechanism:}
  The system shall automatically revert to the last known-good calibration
  if post-update health metrics do not improve or degrade.
  \item \textbf{FR7 – Result Logging:}
  The system shall log all calibration runs, health scores, and parameter changes for evaluation
  and reproducibility.
\end{itemize}

\subsection*{Non-Functional Requirements}
\begin{itemize}
  \item \textbf{NFR1 – Portability:} 
  The calibration framework shall be implemented in Python/C++ with OpenCV
  and remain adaptable to different camera configurations (e.g., Raytrix R5, R25, or similar).
  \item \textbf{NFR2 – Modularity:} 
  Each component (data acquisition, health monitoring, bundle adjustment, logging)
  shall be designed as a separate, reusable module.
  \item \textbf{NFR3 – Reproducibility:} 
  Calibration and monitoring results shall be exportable in a standardized format
  (e.g., JSON) to ensure consistent evaluation.
  \item \textbf{NFR4 – Robustness to Scene Variation:} 
  The calibration process shall tolerate moderate lighting and texture variations
  without significant loss of accuracy (health score drop $\leq10$ points).
\end{itemize}

\end{document}
